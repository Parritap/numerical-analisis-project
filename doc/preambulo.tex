
\section{Objetivo}
Desarrollar e implementar funciones en MATLAB que apliquen los métodos numéricos vistos en clase para resolver problemas de ingeniería, enseñados en la asignatura Análisis numérico y que nos proporciona el docente Edwin Romero Cuero.

\noindent
Buscamos a través de esta actividad, fortalecer la comprensión de los distintos métodos de análisis aprendidos durante este semestre, a su vez comprender los errores de truncamiento, técnicas de búsqueda de raíces, regresiones, interpolación y ecuaciones diferenciales.

% Contenido principal del documento
\section{Introducción}

En esta actividad se desea utilizar varios métodos numéricos y aplicarlos en un sistema de cómputo numérico ``MATLAB'', buscando darle una solución a problemas que son comúnmente vistos en el ámbito de la ingeniería, donde se tratan temas como series de Taylor, métodos de búsqueda de raíces, regresión, interpolación y el uso de los métodos de Euler y Runge-Kutta para resoluciones numéricas de ecuaciones diferenciales. Mediante el uso de los conocimientos apropiados en el ambiente de desarrollo se planea poner en pruebas los conocimientos teóricos con problemas recurrentes en el área laboral.

Los métodos numéricos constituyen técnicas mediante las cuales es posible formular problemas matemáticos de tal forma que puedan resolverse utilizando operaciones aritméticas. Aunque existen muchos tipos de métodos numéricos, todos comparten una característica común: invariablemente se debe realizar un buen número de cálculos aritméticos. No es raro que con el desarrollo de computadoras digitales eficientes y rápidas, el papel de los métodos numéricos en la solución de problemas de ingeniería haya aumentado considerablemente en los últimos años.

\newpage
