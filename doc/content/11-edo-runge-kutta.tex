\section{Integración}

\subsection{Método de Runge-Kutta}

Según \cite{runge-kutta}  en general, si \( k \) es cualquier entero positivo y \( f \) satisface las suposiciones apropiadas, existen métodos numéricos con error de truncamiento local \( O(h^{k+1}) \) para resolver un problema de valor inicial.

\[
	y' = f(x, y), \quad y(x_0) = y_0. \tag{3.3.1}
\]

Además, se puede demostrar que un método con error de truncamiento local \( O(h^{k+1}) \) tiene error de truncamiento global \( O(h^k) \). En las Secciones 3.1 y 3.2 estudiamos métodos numéricos donde \( k = 1 \) y \( k = 2 \). Omitiremos los métodos para los cuales \( k = 3 \) y procederemos al método de Runge-Kutta, el método más ampliamente utilizado, para el cual \( k = 4 \). La magnitud del error de truncamiento local está determinada por la quinta derivada \( y^{(5)} \) de la solución del problema de valor inicial. Por lo tanto, el error de truncamiento local será mayor donde \( |y^{(5)}| \) es grande, o menor donde \( |y^{(5)}| \) es pequeño. El método de Runge-Kutta calcula valores aproximados \( y_1, y_2, ..., y_n \) de la solución de la Ecuación 3.3.1 en \( x_0, x_0 + h, ..., x_0 + nh \) de la siguiente manera: Dado \( y_i \), calcule

\[
	k_{i1} = f(x_i, y_i),
\]

\[
	k_{2i} = f\left( x_i + \frac{h}{2}, y_i + \frac{h}{2} k_{1i} \right),
\]

\[
	k_{3i} = f\left( x_i + \frac{h}{2}, y_i + \frac{h}{2} k_{2i} \right),
\]

\[
	k_{4i} = f(x_i + h, y_i + h k_{3i}),
\]

y

\[
	y_{i+1} = y_i + \frac{h}{6} (k_{1i} + 2k_{2i} + 2k_{3i} + k_{4i}).
\]

\subsubsection{Implementación en MATLAB}

\begin{lstlisting}[caption={Método de Runge-Kutta de cuarto orden}]
% Ejemplo de prueba con bash (Ecuacion: y' = -2*t*y, y(0) = 1, t en [0, 2], h = 0.1):
% cd /Users/esteban/git-repos/numerical-analisis-project && echo -e "-2*t*y\n1\n0\n2\n0.1" | ./run_matlab.sh src/edos/rungeKuttaODE.m

function rungeKuttaODE()
    try
        % Ingrese la ecuacion diferencial en la forma y' = f(t, y)
        syms t y;
        f_str = input('Ingrese la ecuacion diferencial en la forma y'' = f(t, y), f(t, y) = ', 's');
        f = str2func(['@(t, y) ' f_str]);

        % Ingrese las condiciones iniciales
        y0 = input('Ingrese el valor inicial de y (y0): ');
        t0 = input('Ingrese el tiempo inicial (t0): ');
        tf = input('Ingrese el tiempo final (tf): ');
        h = input('Ingrese el tamano del paso (h): ');

        % Calcular el numero de pasos
        numSteps = round((tf - t0) / h);

        % Inicializar arrays para almacenar resultados
        t_values = zeros(1, numSteps + 1);
        y_values = zeros(1, numSteps + 1);

        % Inicializar condiciones iniciales
        t_values(1) = t0;
        y_values(1) = y0;

        % Aplicar el metodo de Runge-Kutta de cuarto orden
        for i = 1:numSteps
            k1 = h * f(t_values(i), y_values(i));
            k2 = h * f(t_values(i) + h/2, y_values(i) + k1/2);
            k3 = h * f(t_values(i) + h/2, y_values(i) + k2/2);
            k4 = h * f(t_values(i) + h, y_values(i) + k3);

            y_values(i + 1) = y_values(i) + (k1 + 2*k2 + 2*k3 + k4) / 6;
            t_values(i + 1) = t_values(i) + h;
        end

        % Mostrar los resultados
        disp('Resultados del metodo de Runge-Kutta de cuarto orden:');
        disp('t_values:');
        disp(t_values);
        disp('y_values:');
        disp(y_values);

        % Graficar la solucion
        plot(t_values, y_values, 'LineWidth', 2);
        title('Solucion de la Ecuacion Diferencial por Runge-Kutta de Cuarto Orden');
        xlabel('t');
        ylabel('y(t)');
        grid on;

    catch
        error('Error en la entrada de datos.');
    end
end
\end{lstlisting}

\subsubsection{Ejemplo de aplicación}

Se resolvió la ecuación diferencial $y' = -2ty$ con condición inicial $y(0) = 1$ en el intervalo $[0, 2]$ usando un tamaño de paso $h = 0.1$.

\textbf{Parámetros:}
\begin{itemize}
	\item Ecuación diferencial: $y' = -2ty$
	\item Condición inicial: $y(0) = 1$
	\item Intervalo: $t \in [0, 2]$
	\item Tamaño de paso: $h = 0.1$
	\item Número de pasos: 20
\end{itemize}

\textbf{Solución analítica:} La ecuación tiene solución exacta $y(t) = e^{-t^2}$, lo que permite verificar la precisión del método numérico.

\textbf{Resultados seleccionados:}

\begin{center}
	\begin{tabular}{|c|c|c|}
		\hline
		\textbf{t} & \textbf{y (Runge-Kutta)} & \textbf{y (Analítica)} \\
		\hline
		0.0        & 1.0000                   & 1.0000                 \\
		0.5        & 0.7788                   & 0.7788                 \\
		1.0        & 0.3679                   & 0.3679                 \\
		1.5        & 0.1054                   & 0.1054                 \\
		2.0        & 0.0183                   & 0.0183                 \\
		\hline
	\end{tabular}
\end{center}

Los resultados muestran una concordancia excelente con la solución analítica, demostrando la alta precisión del método de Runge-Kutta de cuarto orden.

\begin{figure}[H]
	\centering
	\includegraphics[width=0.9\textwidth]{resources/edos/exec-runge-kutta.png}
	\caption{Ejecución del método de Runge-Kutta mostrando los valores de $t$ y $y(t)$ calculados, junto con la gráfica de la solución}
	\label{fig:exec-runge-kutta}
\end{figure}


