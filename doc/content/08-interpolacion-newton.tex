\subsection{Interpolación de Newton}

Según \cite{newtons-interpolation}, como se establece en la introducción, la matriz formada puede estar mal condicionada y ser difícil de invertir. Se puede usar un método más simple para encontrar el polinomio interpolante usando la fórmula de los Polinomios Interpolantes de Newton para ajustar un polinomio de grado $n$ a través de $n+1$ puntos de datos $(x_i, y_i)$ con $1 \leq i \leq n+1$:

\[
	f_n(x) = b_1 + b_2(x-x_1) + b_3(x-x_1)(x-x_2) + \cdots + b_{n+1}(x-x_1)(x-x_2)\cdots(x-x_n)
\]

donde los coeficientes $b_i$ se definen recursivamente usando las diferencias divididas de la siguiente manera:

\begin{align*}
	b_1     & = y_1                                                                                                                               \\
	b_2     & = [y_1, y_2] = \frac{y_2 - y_1}{x_2 - x_1}                                                                                          \\
	b_3     & = [y_1, y_2, y_3] = \frac{[y_2, y_3] - [y_1, y_2]}{x_3 - x_1} = \frac{\frac{y_3-y_2}{x_3-x_2} - \frac{y_2-y_1}{x_2-x_1}}{x_3 - x_1} \\
	        & \vdots                                                                                                                              \\
	b_{n+1} & = [y_1, y_2, y_3, \cdots, y_{n+1}] = \frac{[y_2, y_3, \cdots, y_{n+1}] - [y_1, y_2, y_3, \cdots, y_n]}{x_{n+1} - x_1}
\end{align*}

\subsubsection{Código matlab}

\begin{lstlisting}
% Interpolacion de Newton
    % NOTA!!!!!! -> Es mejor dar la ruta absoluta del CSV para evitar errores
    % cd /Users/esteban/git-repos/numerical-analisis-project && ./run_matlab.sh src/methods/interpolacion/newton.m

    function newtonInterpolation(filename, col_x, col_y, num_points)
        % newtonInterpolation - Realiza interpolacion de Newton sobre datos de un archivo CSV
        %
        % Sintaxis: newtonInterpolation(filename, col_x, col_y, num_points)
        %
        % Argumentos:
        %   filename   - Ruta al archivo CSV (relativa a la raiz del proyecto)
        %   col_x      - Numero de columna para la variable independiente (x)
        %   col_y      - Numero de columna para la variable dependiente (y)
        %   num_points - Numero de puntos a usar para la interpolacion (distribuidos equitativamente)
        %
        % Ejemplo:
        %   newtonInterpolation('guia/datos.csv', 6, 7, 5)
        format long;

        % Obtener el directorio del script y cambiar al directorio del proyecto
        scriptPath = fileparts(mfilename('fullpath'));
        projectPath = fullfile(scriptPath, '..', '..', '..');
        cd(projectPath);
        disp(['Directorio de trabajo: ' pwd]);

        % Leer el archivo CSV
        try
            opts = detectImportOptions(filename, 'Delimiter', ';', 'DecimalSeparator', ',');
            % Leer encabezados primero para mostrar columnas disponibles
            headers = readtable(filename, opts, 'ReadRowNames', false);
            disp('Archivo cargado exitosamente.');
            disp(['Columnas disponibles: ' strjoin(headers.Properties.VariableNames, ', ')]);
            
            % Obtener nombres de las columnas seleccionadas
            column_names = headers.Properties.VariableNames;
            x_label = strrep(column_names{col_x}, '_', ' ');
            y_label = strrep(column_names{col_y}, '_', ' ');
            
            % Ahora leer solo los datos numericos
            data = readmatrix(filename, 'Delimiter', ';', 'DecimalSeparator', ',');
            disp(['Numero de filas: ' num2str(size(data, 1))]);
        catch ME
            disp(['Error: ' ME.message]);
            error('No se pudo leer el archivo. Verifique la ruta y el formato.');
        end

        % Extraer las coordenadas x e y de las columnas especificadas
        x_all = data(:, col_x);
        y_all = data(:, col_y);
        
        % Verificar que los datos sean numericos
        if ~isnumeric(x_all) || ~isnumeric(y_all)
            error('Las columnas seleccionadas deben contener datos numericos.');
        end
        
        % Eliminar filas con valores NaN
        valid_idx = ~isnan(x_all) & ~isnan(y_all);
        x_all = x_all(valid_idx);
        y_all = y_all(valid_idx);

        % Numero total de puntos de datos disponibles
        total_points = length(x_all);
        disp(['Numero de puntos validos en el archivo: ' num2str(total_points)]);
        
        % Validar que haya al menos 2 puntos disponibles
        if total_points <= 1
            error('Se requieren al menos 2 puntos para la interpolacion.');
        end
        
        % Validar que num_points sea valido
        if num_points > total_points
            warning('El numero de puntos solicitado (%d) es mayor que los disponibles (%d). Se usaran todos los puntos.', num_points, total_points);
            num_points = total_points;
        end
        
        if num_points < 2
            error('Se requieren al menos 2 puntos para la interpolacion.');
        end
        
        % Seleccionar num_points distribuidos equitativamente
        if num_points == total_points
            % Usar todos los puntos
            selected_indices = 1:total_points;
        else
            % Distribuir equitativamente los indices
            selected_indices = round(linspace(1, total_points, num_points));
            % Asegurar que sean unicos (por si hay redondeos duplicados)
            selected_indices = unique(selected_indices);
        end
        
        % Extraer los puntos seleccionados para la interpolacion
        x_data = x_all(selected_indices);
        y_data = y_all(selected_indices);
        
        % Numero final de puntos a usar
        n = length(x_data);
        disp(['Numero de puntos seleccionados para interpolacion: ' num2str(n)]);
        disp(['Rango de x: [' num2str(min(x_data)) ', ' num2str(max(x_data)) ']']);

        % Crear una tabla de diferencias divididas
        f = zeros(n, n);
        f(:,1) = y_data;

        for j = 2:n
            for i = 1:n-j+1
                f(i,j) = (f(i+1,j-1) - f(i,j-1)) / (x_data(i+j-1) - x_data(i));
            end
        end

        % Construir el polinomio de Newton simbolicamente
        syms x;
        newtonPoly = f(1,1);
        for j = 2:n
            producto = 1;
            for i = 1:j-1
                producto = producto * (x - x_data(i));
            end
            newtonPoly = newtonPoly + f(1,j) * producto;
        end
        
        % Simplificar y mostrar el polinomio
        newtonPoly = simplify(newtonPoly);
        fprintf('\033[32mEl polinomio interpolador de Newton (grado %d) es:\n%s\033[0m\n', n-1, char(newtonPoly));
        
        % Generar puntos para graficar la curva ajustada
        x_fit = linspace(min(x_data), max(x_data), 300);
        y_fit = double(subs(newtonPoly, x, x_fit));
        
        % Calcular limites del eje Y basados en los datos
        y_min = min([y_all; y_fit(:)]);
        y_max = max([y_all; y_fit(:)]);
        y_range = y_max - y_min;
        
        % Identificar puntos no seleccionados
        not_selected_mask = true(size(x_all));
        not_selected_mask(selected_indices) = false;
        
        % Graficar el polinomio primero (linea verde)
        plot(x_fit, y_fit, 'g-', 'LineWidth', 2.5);
        hold on;
        
        % Graficar puntos NO seleccionados (si existen) en gris
        if any(not_selected_mask)
            scatter(x_all(not_selected_mask), y_all(not_selected_mask), 50, [0.7 0.7 0.7], 'filled', 'MarkerFaceAlpha', 0.4);
        end
        
        % Graficar puntos seleccionados en verde
        scatter(x_data, y_data, 80, 'g', 'filled', 'MarkerEdgeColor', 'k', 'LineWidth', 1.5, 'MarkerFaceAlpha', 0.7);
        
        % Agregar etiquetas y formato al grafico usando los nombres de las columnas
        xlabel(x_label, 'FontSize', 12, 'FontWeight', 'bold');
        ylabel(y_label, 'FontSize', 12, 'FontWeight', 'bold');
        title(['Interpolacion de Newton: ' y_label ' vs ' x_label ' (n=' num2str(n) ')'], 'FontSize', 14, 'FontWeight', 'bold');
        
        % Leyenda dinamica
        if any(not_selected_mask)
            legend('Polinomio Interpolador', 'Datos no seleccionados', 'Puntos seleccionados', 'Location', 'best');
        else
            legend('Polinomio Interpolador', 'Puntos seleccionados', 'Location', 'best');
        end
        
        grid on;
        ylim([y_min - 0.1*y_range, y_max + 0.1*y_range]);
        hold off;
    end
    newtonInterpolation('/Users/esteban/git-repos/numerical-analisis-project/guia/datos.csv', 6, 7, 5)
\end{lstlisting}

\subsubsection{Ejemplo de uso con dataset de prueba}

Con base al dataset mostrado en la figura \ref{fig:dataset}, podemos hacer uso del algoritmo de interpolación de Newton para obtener un polinomio que pase exactamente por ciertos puntos seleccionados.

Para el presente ejemplo, seleccionaremos 5 puntos distribuidos equitativamente del dataset en las columnas 6 y 7 (Nota Algoritmos VS Horas de estudio). El resultado del polinomio computado es, como se muestra en la figura \ref{fig:exec:newton}, como sigue.

\[
	P(x) = \frac{2673199x}{15400} - \frac{118099x^2}{4400} + \frac{53513x^3}{30800} - \frac{619x^4}{15400} - \frac{296267}{770}
\]

\begin{figure}[H]
	\centering
	\includegraphics[scale=0.38]{resources/interpolacion/exec-newton.png}
	\caption{Ejecución del algoritmo de interpolación de Newton con 5 puntos seleccionados del dataset de prueba}
	\label{fig:exec:newton}
\end{figure}

La gráfica del polinomio interpolador se puede observar en la figura \ref{fig:plot:newton}.

\begin{figure}[H]
	\centering
	\includegraphics[scale=0.4]{resources/interpolacion/plot-newton.png}
	\caption{Gráfica del polinomio interpolador de Newton con 5 puntos seleccionados del dataset de prueba}
	\label{fig:plot:newton}
\end{figure}
