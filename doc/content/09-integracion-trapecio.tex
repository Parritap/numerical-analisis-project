\section{Integración Numérica}

\subsection{Método del Trapecio}


Según \cite{Stewart2012_Una}, la fórmula para aproximar la integral definida de una función $f$ en el intervalo $[a, b]$ usando el método del trapecio es la siguiente:

\[
	\int_a^b f(x) \, dx \approx T_n = \frac{\Delta x}{2} \left[ f(x_0) + 2f(x_1) + 2f(x_2) + \cdots + 2f(x_{n-1}) + f(x_n) \right]
\]

donde $\Delta x = (b - a)/n$ y $x_i = a + i \, \Delta x$.

Para datos discretos, la regla del trapecio se aplica calculando el área de cada trapecio individual formado entre puntos consecutivos:

\[
	\text{Área} \approx \sum_{i=1}^{n-1} \frac{(x_{i+1} - x_i) \cdot (y_i + y_{i+1})}{2}
\]

donde cada término representa el área de un trapecio con bases $y_i$ y $y_{i+1}$, y altura $(x_{i+1} - x_i)$.

\subsubsection{Implementación en MATLAB}
\begin{lstlisting}[caption={Implementación del método del trapecio}]
% Aplicar la Regla del Trapecio
% Formula: Integral aprox= sum[(x[i+1] - x[i]) * (y[i] + y[i+1]) / 2]
integral = 0.0;
detalles = cell(n-1, 1);

for i = 1:(n-1)
    h = x_data(i+1) - x_data(i);  % Ancho del intervalo
    area_parcial = h * (y_data(i) + y_data(i+1)) / 2;
    integral = integral + area_parcial;
    
    % Guardar detalles de cada trapecio
    detalles{i} = struct('intervalo', ...
        sprintf('[%.2f, %.2f]', x_data(i), x_data(i+1)), ...
        'h', h, 'area', area_parcial);
end
\end{lstlisting}

\textbf{Función de visualización:}

\begin{lstlisting}[caption={Visualización de trapecios}]
% Graficar los datos y los trapecios
figure('Position', [100, 100, 1200, 600]);
hold on;

% Dibujar los trapecios primero (para que esten detras)
for i = 1:(n-1)
    % Coordenadas de los vertices del trapecio
    x_trap = [x_data(i), x_data(i+1), x_data(i+1), x_data(i)];
    y_trap = [0, 0, y_data(i+1), y_data(i)];
    
    % Rellenar el trapecio con color cian y borde azul
    fill(x_trap, y_trap, 'cyan', 'FaceAlpha', 0.3, ...
         'EdgeColor', 'blue', 'LineWidth', 1);
end

% Graficar la linea que conecta los puntos
plot(x_data, y_data, 'b-', 'LineWidth', 2, 'DisplayName', 'Datos');

% Graficar los puntos de datos
scatter(x_data, y_data, 80, 'b', 'filled', ...
        'MarkerEdgeColor', 'k', 'LineWidth', 1, ...
        'DisplayName', 'Puntos de datos');

xlabel(x_label, 'FontSize', 12, 'FontWeight', 'bold');
ylabel(y_label, 'FontSize', 12, 'FontWeight', 'bold');
title(sprintf('Regla del Trapecio: %s vs %s\nArea = %.4f', ...
              y_label, x_label, integral), ...
      'FontSize', 14, 'FontWeight', 'bold');

legend('Trapecios', 'Datos', 'Puntos de datos', 'Location', 'best');
grid on;
hold off;
\end{lstlisting}

\subsubsection{Resultados}

Se aplicó el método del trapecio a los datos académicos, específicamente para calcular el rendimiento académico acumulado entre las variables \textbf{Horas de Estudio} (X) y \textbf{Nota en Algoritmos} (Y).

\textbf{Parámetros de ejecución:}
\begin{itemize}
	\item \textbf{Variables:} Horas Estudio (X) vs Nota Algoritmos (Y)
	\item \textbf{Rango de integración:} [5.00, 20.00]
	\item \textbf{Número de intervalos (trapecios):} 29
	\item \textbf{Número de puntos:} 30
\end{itemize}

\textbf{Resultado principal:}

\begin{center}
	\colorbox{green!20}{\textbf{Área bajo la curva (Integral): 53.850000}}
\end{center}

\textbf{Detalles de los primeros 5 trapecios:}

\begin{center}
	\begin{tabular}{|l|c|c|}
		\hline
		\textbf{Intervalo} & \textbf{Ancho (h)} & \textbf{Área Parcial} \\
		\hline
		{[5.00, 5.00]}     & 0.0000             & 0.000000              \\
		{[5.00, 5.00]}     & 0.0000             & 0.000000              \\
		{[5.00, 5.00]}     & 0.0000             & 0.000000              \\
		{[5.00, 6.00]}     & 1.0000             & 3.150000              \\
		{[6.00, 7.00]}     & 1.0000             & 3.150000              \\
		\hline
		\multicolumn{3}{|c|}{\textit{... (24 trapecios más)}}           \\
		\hline
	\end{tabular}
\end{center}

\textbf{Salida de consola:}

\begin{figure}[H]
	\centering
	\includegraphics[width=\textwidth]{resources/integracion/exec-trapezoidal.png}
	\caption{Salida de ejecución del método del trapecio mostrando los detalles del cálculo}
	\label{fig:exec-trapezoidal}
\end{figure}

\subsubsection{Visualización}

La Figura \ref{fig:plot-trapezoidal} muestra la representación gráfica del método del trapecio aplicado a los datos. Los trapecios en color cian representan las áreas parciales que se suman para obtener el área total bajo la curva. La línea azul conecta los puntos de datos, y cada punto está marcado para facilitar la visualización.

\begin{figure}[H]
	\centering
	\includegraphics[width=\textwidth]{resources/integracion/plot-trapezoidal.png}
	\caption{Visualización del método del trapecio aplicado a la relación entre Horas de Estudio y Nota en Algoritmos. Los trapecios sombreados muestran las áreas parciales que se suman para calcular el área total de 53.8500}
	\label{fig:plot-trapezoidal}
\end{figure}

