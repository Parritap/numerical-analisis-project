\subsection{Método de Newton-Raphson}

El método de Newton-Raphson es un algoritmo iterativo para encontrar raíces de funciones reales. Utiliza la derivada de la función para aproximar la raíz a partir de un valor inicial \cite{newtons-method}. Según \cite{newton-optimization} el método de Newton es una de las herramientas fundamentales en análisis numérico, investigación de operaciones, optimización y control. Tiene numerosas aplicaciones en la ciencia de la administración, la investigación industrial y financiera, y la minería de datos.

\subsubsection{Fundamento teórico}

El método se basa en la aproximación lineal de la función mediante su serie de Taylor. Si $x_n$ es una aproximación a la raíz, entonces la siguiente aproximación $x_{n+1}$ se obtiene encontrando donde la línea tangente a la curva en el punto $(x_n, f(x_n))$ cruza el eje $x$.

\begin{figure}
	\centering
	\includegraphics[scale=0.4]{resources/newton-root-plot.png}
\end{figure}

\subsubsection{Fórmula iterativa}

La fórmula de Newton-Raphson es:
\begin{equation}
	x_{n+1} = x_n - \frac{f(x_n)}{f'(x_n)}
\end{equation}

Geométricamente, esto representa extender la tangente a la curva en el punto $(x_n, f(x_n))$ hasta que intersecte el eje $x$.

\subsubsection{Algoritmo}

\begin{enumerate}
	\item Elegir un valor inicial $x_0$ cercano a la raíz esperada
	\item Calcular $f(x_n)$ y $f'(x_n)$
	\item Verificar que $f'(x_n) \neq 0$
	\item Calcular la siguiente aproximación: $x_{n+1} = x_n - \frac{f(x_n)}{f'(x_n)}$
	\item Verificar el criterio de convergencia: $|x_{n+1} - x_n| < \epsilon$ o $|f(x_{n+1})| < \epsilon$
	\item Si no se cumple el criterio, repetir desde el paso 2
\end{enumerate}

\subsubsection{Ventajas y desventajas}

\textbf{Ventajas:}
\begin{itemize}
	\item Convergencia cuadrática cuando se está cerca de la raíz
	\item Requiere menos iteraciones que bisección
	\item Muy preciso cuando converge
\end{itemize}

\textbf{Desventajas:}
\begin{itemize}
	\item Requiere calcular la derivada $f'(x)$
	\item No garantiza convergencia si la elección inicial es mala
	\item Puede diverger o ciclar si $f'(x)$ es muy pequeña
	\item Sensible a la elección del punto inicial
\end{itemize}

\subsubsection{Condiciones de convergencia}

El método converge si:
\begin{itemize}
	\item La función es continuamente diferenciable
	\item La derivada no se anula en la vecindad de la raíz
	\item El valor inicial $x_0$ está suficientemente cerca de la raíz
\end{itemize}

