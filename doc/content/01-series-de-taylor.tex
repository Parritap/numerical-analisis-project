\section{Aproximaciones por Series de Taylor}
\subsection{Series de Taylor}



\subsubsection{Fundamento Teórico}

Según \cite{Romero2019}, cuando se habla del resto de Taylor, nos referimos al error generado cuando aproximamos los valores de una función no polinómica a través de un polinomio.


La ecuación de una serie de Taylor nos la da \cite{Stewart2012_Una}, la cual se ve en la \hbox{ecuación \ref{eq:taylor-general}.}


\begin{equation}
	\boxed{
		\begin{aligned}
			f(x) & = \sum_{n=0}^{\infty} \frac{f^{(n)}(a)}{n!} (x - a)^n                                                   \\[1em]
			     & = f(a) + \frac{f'(a)}{1!} (x - a) + \frac{f''(a)}{2!} (x - a)^2 + \frac{f'''(a)}{3!} (x - a)^3 + \cdots
		\end{aligned}
	}
	\label{eq:taylor-general}
\end{equation}

Para el caso especial $a = 0$ la ecuación cambia un poco, y se le atribuye el nombre de \textbf{series de MacLaurin} \citep{Stewart2012_Una}.

\begin{equation}
	\boxed{
		f(x) = \sum_{n=0}^{\infty} \frac{f^{(n)}(0)}{n!} x^n = f(0) + \frac{f'(0)}{1!} x + \frac{f''(0)}{2!} x^2 + \cdots
	}
	\label{eq:mclaurin}
\end{equation}
