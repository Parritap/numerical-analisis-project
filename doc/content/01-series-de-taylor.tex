\section{Aproximaciones por Series de Taylor}
\subsection{Series de Taylor}



\subsubsection{Fundamento Teórico}

Según \cite{Romero2019}, cuando se habla del resto de Taylor, nos referimos al error generado cuando aproximamos los valores de una función no polinómica a través de un polinomio.


La ecuación de una serie de Taylor nos la da \cite{Stewart2012_Una}, la cual se ve en la \hbox{ecuación \ref{eq:taylor-general}.}


\begin{equation}
	\boxed{
		\begin{aligned}
			f(x) & = \sum_{n=0}^{\infty} \frac{f^{(n)}(a)}{n!} (x - a)^n                                                   \\[1em]
			     & = f(a) + \frac{f'(a)}{1!} (x - a) + \frac{f''(a)}{2!} (x - a)^2 + \frac{f'''(a)}{3!} (x - a)^3 + \cdots
		\end{aligned}
	}
	\label{eq:taylor-general}
\end{equation}

Para el caso especial $a = 0$ la ecuación cambia un poco, y se le atribuye el nombre de \textbf{series de MacLaurin} \citep{Stewart2012_Una}.

\begin{equation}
	\boxed{
		f(x) = \sum_{n=0}^{\infty} \frac{f^{(n)}(0)}{n!} x^n = f(0) + \frac{f'(0)}{1!} x + \frac{f''(0)}{2!} x^2 + \cdots
	}
	\label{eq:mclaurin}
\end{equation}

\subsubsection{Aproximación de Orden N}

En la práctica, no es posible calcular una serie infinita de términos. Por lo tanto, se utiliza una aproximación de orden $N$, conocida como el \textbf{polinomio de Taylor de orden N} \citep{taylor_rosario_2013}:

\begin{equation}
	\boxed{
		T_N(x) = \sum_{n=0}^{N} \frac{f^{(n)}(a)}{n!} (x - a)^n
	}
	\label{eq:taylor-orden-n}
\end{equation}

El error de esta aproximación está dado por el \textbf{resto de Taylor} o \textbf{término del error}, que representa la diferencia entre la función real y su aproximación polinómica:

\begin{equation}
	R_N(x) = f(x) - T_N(x)
	\label{eq:resto-taylor}
\end{equation}

\subsubsection{Ejemplos Comunes de Series de Taylor}

Algunas funciones tienen series de Taylor bien conocidas alrededor de $a = 0$ (series de MacLaurin) \citep{Stewart2012_Una}:

\begin{align}
	e^x      & = 1 + x + \frac{x^2}{2!} + \frac{x^3}{3!} + \frac{x^4}{4!} + \cdots = \sum_{n=0}^{\infty} \frac{x^n}{n!}                    \label{eq:taylor-exp} \\[1em]
	\sin(x)  & = x - \frac{x^3}{3!} + \frac{x^5}{5!} - \frac{x^7}{7!} + \cdots = \sum_{n=0}^{\infty} \frac{(-1)^n x^{2n+1}}{(2n+1)!}      \label{eq:taylor-sin}  \\[1em]
	\cos(x)  & = 1 - \frac{x^2}{2!} + \frac{x^4}{4!} - \frac{x^6}{6!} + \cdots = \sum_{n=0}^{\infty} \frac{(-1)^n x^{2n}}{(2n)!}          \label{eq:taylor-cos}  \\[1em]
	\ln(1+x) & = x - \frac{x^2}{2} + \frac{x^3}{3} - \frac{x^4}{4} + \cdots = \sum_{n=1}^{\infty} \frac{(-1)^{n+1} x^n}{n} \quad (|x| < 1) \label{eq:taylor-ln}
\end{align}

\subsubsection{Implementación Computacional}

El algoritmo implementado para calcular una serie de Taylor de orden $N$ alrededor del punto $a$ utiliza un enfoque iterativo que calcula las derivadas de forma incremental. El procedimiento es el siguiente:

\begin{enumerate}
	\item Inicializar la serie de Taylor: $T(x) = 0$
	\item Inicializar la función derivada: $f_{\text{deriv}} = f(x)$ (derivada de orden 0)
	\item Para cada término $n = 0, 1, 2, \ldots, N$:
	      \begin{enumerate}
		      \item Evaluar la derivada actual en el punto $a$: $f^{(n)}(a) = f_{\text{deriv}}(a)$
		      \item Calcular el coeficiente: $c_n = \frac{f^{(n)}(a)}{n!}$
		      \item Calcular el término actual:
		            \[
			            t_n = \begin{cases}
				            c_n                 & \text{si } n = 0 \\
				            c_n \cdot (x - a)^n & \text{si } n > 0
			            \end{cases}
		            \]
		      \item Agregar el término a la serie: $T(x) = T(x) + t_n$
		      \item Si $n < N$, calcular la siguiente derivada: $f_{\text{deriv}} = \frac{d}{dx}f_{\text{deriv}}$
	      \end{enumerate}
	\item Simplificar la expresión final: $T_N(x) = \text{simplify}(T(x))$
\end{enumerate}

Este enfoque iterativo es eficiente ya que calcula cada derivada simbólicamente a partir de la anterior, evitando recalcular todas las derivadas desde cero. La implementación utiliza cálculo simbólico de MATLAB para obtener expresiones exactas de las derivadas y del polinomio resultante.

\subsubsection{Código Matlab}

\begin{lstlisting}
format long;
syms x;
sympref("PolynomialDisplayStyle","ascend"); % Imprimir en orden ascendente, es decir, de menor a mayor grado.




% Solicitar entrada al usuario
f_str = input('Ingrese la funcion: f(x) = ', 's');
FUN = str2sym(f_str);

a = input('Ingrese el punto central de la serie (a): ');
N = input('Ingrese el orden de la serie de Taylor (N): ');

% Validar que N sea un entero positivo
if N < 0 || floor(N) ~= N
    error('El orden N debe ser un entero no negativo.');
end

% Obtener la variable simbolica
x_var = symvar(FUN);
if isempty(x_var)
    error('La funcion debe contener al menos una variable.');
end
x = x_var(1);

% Inicializar la serie de Taylor
T = sym(0);
f_deriv = FUN;

% Imprimir encabezado
fprintf('\n%s\n', repmat('=', 1, 90));
fprintf('CALCULO DE SERIE DE TAYLOR\n');
fprintf('%s\n', repmat('=', 1, 90));
fprintf('Funcion: %s\n', char(FUN));
fprintf('Punto central (a): %s\n', char(a));
fprintf('Orden (N): %d\n', N);
fprintf('%s\n', repmat('=', 1, 90));

% Imprimir tabla de terminos
fprintf('\n%-10s %-30s %-30s %-30s\n', 'Termino', 'f^(n)(a)', 'Coeficiente [ f^(n)(a)/n! ] ', 'Termino de la serie');
fprintf('%s\n', repmat('-', 1, 105));

% Calcular la serie: sum_{n=0}^{N} f^(n)(a)/n! * (x-a)^n
for n = 0:N
        % Evaluar la n-esima derivada en el punto a
        f_at_a = subs(f_deriv, x, a);

        % Calcular el coeficiente
    coef = f_at_a / factorial(n);

    % Calcular el termino actual
    if n == 0
        term = coef;
    else
        term = coef * (x - a)^n;
    end

    % Agregar el termino a la serie
    T = T + term;

    % Imprimir informacion del termino
    fprintf('%-10d %-30s %-30s %-30s\n', ...
        n, ...
        char(vpa(f_at_a, 6)), ...
        char(vpa(coef, 6)), ...
        char(term));

    % Calcular la siguiente derivada para la proxima iteracion
    if n < N
        f_deriv = diff(f_deriv, x);
    end
end

fprintf('%s\n', repmat('-', 1, 105));

% Simplificar la serie final
T = simplify(T);

% Mostrar resultado final
fprintf('\n%s\n', repmat('=', 1, 90));
fprintf('RESULTADO FINAL\n');
fprintf('%s\n', repmat('=', 1, 90));
fprintf('Serie de Taylor de orden %d:\n\n', N);
fprintf('\x1b[32mT_%d(x) = %s\x1b[0m\n', N, char(T));
fprintf('\n%s\n', repmat('=', 1, 90));
\end{lstlisting}

\newpage
\subsubsection{Probando ecuaciones conocidas}

\textbf{Probando con $e^x$ centrado en 0 y de grado 5}: el resultado de la ejecución puede verse en la figura \ref{fig:taylor-exp}, así como su graphica en la figura \ref{fig:graph-taylor-exp}

\begin{figure}[H]
	\centering
	\includegraphics[scale=0.35]{resources/01-taylor/01-exe-exp(x).png}
	\caption{Serie de Taylor de $e^x$ alrededor de $x=0$ con orden 5}
	\label{fig:taylor-exp}
\end{figure}

\begin{figure}[H]
	\centering
	\includegraphics[scale=0.6]{resources/01-taylor/01-graph-exp(x).png}
	\caption{Comparación entre la serie de Taylor alrededor de $0$ de orden 5 y la función original $e^x$}
	\label{fig:graph-taylor-exp}
\end{figure}

\textbf{Probando con $sin(2x)$ centrado en $\frac{\pi}{2}$ y de grado 8}:

\begin{figure}[H]
	\centering
	\includegraphics[scale=0.35]{resources/01-taylor/02-exec-sin(2x).png}
	\caption{Serie de Taylor de $e^x$ alrededor de $x=0$ con orden 5}
	\label{fig:taylor-sin-exec}
\end{figure}

\begin{figure}[H]
	\centering
	\includegraphics[scale=0.6]{resources/01-taylor/02-graph-sin(2x).png}
	\caption{Comparación entre la serie de Taylor alrededor de $0$ de orden 5 y la función original $e^x$}
	\label{fig:taylor-sin-graph}
\end{figure}

