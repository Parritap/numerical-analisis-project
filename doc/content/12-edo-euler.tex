\section{Método de Euler}

El método de Euler es el método numérico más simple para resolver ecuaciones diferenciales ordinarias. Aunque es menos preciso que métodos de orden superior como Runge-Kutta, su simplicidad lo convierte en una herramienta didáctica fundamental para comprender los métodos numéricos.

\subsection{Formulación del método}

Para resolver el problema de valor inicial:

\[
	y' = f(t, y), \quad y(t_0) = y_0
\]

El método de Euler aproxima la solución mediante la fórmula iterativa:

\[
	y_{i+1} = y_i + h \cdot f(t_i, y_i)
\]

donde:
\begin{itemize}
	\item $h$ es el tamaño del paso
	\item $t_{i+1} = t_i + h$
	\item $f(t_i, y_i)$ es la pendiente en el punto $(t_i, y_i)$
\end{itemize}

El método tiene error de truncamiento local $O(h^2)$ y error global $O(h)$, lo que significa que es un método de primer orden.

\subsection{Implementación en MATLAB}

\begin{lstlisting}[caption={Método de Euler}]
% Ejemplo de prueba con bash (Ecuacion: y' = -2*t*y, y(0) = 1, t en [0, 2], h = 0.1):
% cd /Users/esteban/git-repos/numerical-analisis-project && echo -e "-2*t*y\n1\n0\n2\n0.1" | ./run_matlab.sh src/edos/eulerODE.m

function eulerODE()
    try
        % Ingrese la ecuacion diferencial en la forma y' = f(t, y)
        syms t y;
        f_str = input('Ingrese la ecuacion diferencial en la forma y'' = f(t, y), f(t, y) = ', 's');
        f = str2func(['@(t, y) ' f_str]);

        % Ingrese las condiciones iniciales
        y0 = input('Ingrese el valor inicial de y (y0): ');
        t0 = input('Ingrese el tiempo inicial (t0): ');
        tf = input('Ingrese el tiempo final (tf): ');
        h = input('Ingrese el tamano del paso (h): ');

        % Calcular el numero de pasos
        numSteps = round((tf - t0) / h);

        % Inicializar arrays para almacenar resultados
        t_values = zeros(1, numSteps + 1);
        y_values = zeros(1, numSteps + 1);

        % Inicializar condiciones iniciales
        t_values(1) = t0;
        y_values(1) = y0;

        % Aplicar el metodo de Euler
        for i = 1:numSteps
            y_values(i + 1) = y_values(i) + h * f(t_values(i), y_values(i));
            t_values(i + 1) = t_values(i) + h;
        end

        % Mostrar los resultados
        disp('Resultados del metodo de Euler:');
        disp('t_values:');
        disp(t_values);
        disp('y_values:');
        disp(y_values);

        % Graficar la solucion
        plot(t_values, y_values, 'LineWidth', 2, 'Color', 'b');
        title('Solucion de la Ecuacion Diferencial por Metodo de Euler');
        xlabel('t');
        ylabel('y(t)');
        grid on;

    catch
        error('Error en la entrada de datos.');
    end
end
\end{lstlisting}

\subsection{Ejemplo de aplicación}

Se resolvió la ecuación diferencial $y' = -2ty$ con condición inicial $y(0) = 1$ en el intervalo $[0, 2]$ usando un tamaño de paso $h = 0.1$.

\textbf{Parámetros:}
\begin{itemize}
	\item Ecuación diferencial: $y' = -2ty$
	\item Condición inicial: $y(0) = 1$
	\item Intervalo: $t \in [0, 2]$
	\item Tamaño de paso: $h = 0.1$
	\item Número de pasos: 20
\end{itemize}

\textbf{Solución analítica:} $y(t) = e^{-t^2}$

\textbf{Comparación de resultados:}

\begin{center}
	\begin{tabular}{|c|c|c|c|}
		\hline
		\textbf{t} & \textbf{Euler} & \textbf{Runge-Kutta} & \textbf{Analítica} \\
		\hline
		0.0        & 1.0000         & 1.0000               & 1.0000             \\
		0.5        & 0.7713         & 0.7788               & 0.7788             \\
		1.0        & 0.3606         & 0.3679               & 0.3679             \\
		1.5        & 0.1006         & 0.1054               & 0.1054             \\
		2.0        & 0.0171         & 0.0183               & 0.0183             \\
		\hline
	\end{tabular}
\end{center}

Los resultados muestran que el método de Euler proporciona aproximaciones razonables, aunque con menor precisión que Runge-Kutta. El error se acumula conforme avanza la integración.

\begin{figure}[H]
	\centering
	\includegraphics[width=0.9\textwidth]{resources/edos/exec-euler.png}
	\caption{Ejecución del método de Euler mostrando los valores de $t$ y $y(t)$ calculados, junto con la gráfica de la solución}
	\label{fig:exec-euler}
\end{figure}

\subsection{Análisis comparativo}

\textbf{Ventajas del método de Euler:}
\begin{itemize}
	\item Implementación simple y directa
	\item Bajo costo computacional por paso
	\item Excelente para fines didácticos
	\item Útil para problemas donde se requiere rapidez sobre precisión
\end{itemize}

\textbf{Desventajas:}
\begin{itemize}
	\item Error de truncamiento $O(h)$ requiere pasos pequeños para buena precisión
	\item Acumulación de errores en integraciones largas
	\item Menos estable que métodos de orden superior
	\item No recomendado para problemas rígidos (stiff)
\end{itemize}

\textbf{Comparación con Runge-Kutta:}
\begin{itemize}
	\item Euler: 1 evaluación de $f$ por paso, error $O(h)$
	\item Runge-Kutta: 4 evaluaciones de $f$ por paso, error $O(h^4)$
	\item Para la misma precisión, Euler requiere pasos mucho más pequeños
	\item Runge-Kutta es generalmente más eficiente para problemas que requieren alta precisión
\end{itemize}
