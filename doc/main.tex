\documentclass[12pt,a4paper]{article}

% Paquetes para el idioma español
\usepackage[spanish,es-tabla]{babel}

%% Configuración de fuente Arial
%\usepackage{fontspec}
%\setmainfont{Arial}

% Paquetes adicionales
\usepackage{xcolor}
\usepackage{geometry}
\usepackage{setspace}
\usepackage{graphicx}
\usepackage{amsmath}
\usepackage{amsfonts}
\usepackage{amssymb}
\usepackage{url}
\usepackage{hyperref}
\usepackage[authoryear,round]{natbib}
\usepackage[table]{xcolor}
\usepackage{colortbl}
\usepackage{array}
\usepackage{longtable}
\usepackage{float}
\usepackage{caption}
\usepackage{listings}
\usepackage{algorithm}
\usepackage{algpseudocode}
\usepackage{tikz}
\usetikzlibrary{babel}
\usepackage{pgfplots}
\pgfplotsset{compat=1.18}


\definecolor{mygreen}{RGB}{50, 140, 105}

% Lset config for Matlab code
\lstset{
    language=Matlab,
    basicstyle=\scriptsize\ttfamily,
    keywordstyle=\color{blue}\bfseries,
    commentstyle=\color{green!60!black}\itshape,
    stringstyle=\color{red},
    numberstyle=\tiny\color{gray},
    numbers=left,
    numbersep=8pt,
    stepnumber=1,
    frame=single,
    breaklines=true,
    breakatwhitespace=true,
    tabsize=4,
    showstringspaces=false,
    xleftmargin=15pt,
    xrightmargin=5pt,
    backgroundcolor=\color{gray!5}
}

% Definir colores para tablas
\definecolor{HeaderGreen}{RGB}{23,162,153}
\definecolor{LightGreen}{RGB}{204,227,220}

% Configuración de natbib para español
\providecommand{\andname}{and}
\renewcommand{\andname}{y}
\addto\captionsspanish{%
  \renewcommand{\bibname}{Bibliograf\'ia}%
  \renewcommand{\refname}{Referencias}%
}

% Configuración de márgenes
\geometry{
    left=2.5cm,
    right=2.5cm,
    top=2.5cm,
    bottom=2.5cm
}

% Configuración de interlineado
\onehalfspacing

% Configuración de hipervínculos
\hypersetup{
    colorlinks=true,
    linkcolor=black,
    citecolor=blue,
    urlcolor=blue,
    pdftitle={Métodos numéricos en Matlab},
    pdfauthor={Juan Esteban Parra Parra},
    pdfsubject={Análisis Numérico},
    pdfkeywords={métodos numéricos, MATLAB, raíces, regresión}
}

% Elimina el espacio al inicio de cada párrafo
\setlength{\parindent}{0pt}
% Separa los parrafos ya que no existe espacio al inicio de cada párrafo
\setlength{\parskip}{8pt}

% Información del documento
\title{Métodos numéricos en Matlab}
\author{Juan Esteban Parra Parra}
\date{\today}

\begin{document}

% Página de título
\begin{titlepage}
	\centering

	\vspace{2em}
	{\textbf{Métodos numéricos en Matlab}}

	{Profesor:}\\
	\vspace{0.1cm}
	{Edwin Romero Cuero}\\

	\vspace{1.2cm}

	% Logo centrado como primer elemento
	\begin{center}
		\includegraphics[width=0.4\textwidth]{resources/uniquindio.png}
	\end{center}

	\vspace{1.1cm}
	% ####### PRESENTACION
	{\textbf{Presentado por:}}\\
	\vspace{0.1cm}
	{Juan Esteban Parra Parra}\\

	\vfill{
		\vspace{1cm}
		\vspace{1cm}

		{Universidad del Quindío --- Facultad de ingeniería\\Programa de Ingeniería de Sistemas y Computación}\\
		\vspace{0.5cm}
		{Armenia -- Quindío 2025 }
	}

\end{titlepage}

% Tabla de contenidos
\newpage
\tableofcontents
\newpage


% ##########################################################################
% ##########################   CONTENIDO ###################################
% ##########################################################################

% Objetivo e introducción

\section{Objetivo}
Desarrollar e implementar funciones en MATLAB que apliquen los métodos numéricos vistos en clase para resolver problemas de ingeniería, enseñados en la asignatura Análisis numérico y que nos proporciona el docente Edwin Romero Cuero.

\noindent
Buscamos a través de esta actividad, fortalecer la comprensión de los distintos métodos de análisis aprendidos durante este semestre, a su vez comprender los errores de truncamiento, técnicas de búsqueda de raíces, regresiones, interpolación y ecuaciones diferenciales.

% Contenido principal del documento
\section{Introducción}

En esta actividad se desea utilizar varios métodos numéricos y aplicarlos en un sistema de cómputo numérico ``MATLAB'', buscando darle una solución a problemas que son comúnmente vistos en el ámbito de la ingeniería, donde se tratan temas como series de Taylor, métodos de búsqueda de raíces, regresión, interpolación y el uso de los métodos de Euler y Runge-Kutta para resoluciones numéricas de ecuaciones diferenciales. Mediante el uso de los conocimientos apropiados en el ambiente de desarrollo se planea poner en pruebas los conocimientos teóricos con problemas recurrentes en el área laboral.

Los métodos numéricos constituyen técnicas mediante las cuales es posible formular problemas matemáticos de tal forma que puedan resolverse utilizando operaciones aritméticas. Aunque existen muchos tipos de métodos numéricos, todos comparten una característica común: invariablemente se debe realizar un buen número de cálculos aritméticos. No es raro que con el desarrollo de computadoras digitales eficientes y rápidas, el papel de los métodos numéricos en la solución de problemas de ingeniería haya aumentado considerablemente en los últimos años.

\newpage


% BUSQUEDA DE RAICES
\section{Aproximaciones por Series de Taylor}
\subsection{Series de Taylor}



\subsubsection{Fundamento Teórico}

Según \cite{Romero2019}, cuando se habla del resto de Taylor, nos referimos al error generado cuando aproximamos los valores de una función no polinómica a través de un polinomio.


La ecuación de una serie de Taylor nos la da \cite{Stewart2012_Una}, la cual se ve en la \hbox{ecuación \ref{eq:taylor-general}.}


\begin{equation}
	\boxed{
		\begin{aligned}
			f(x) & = \sum_{n=0}^{\infty} \frac{f^{(n)}(a)}{n!} (x - a)^n                                                   \\[1em]
			     & = f(a) + \frac{f'(a)}{1!} (x - a) + \frac{f''(a)}{2!} (x - a)^2 + \frac{f'''(a)}{3!} (x - a)^3 + \cdots
		\end{aligned}
	}
	\label{eq:taylor-general}
\end{equation}

Para el caso especial $a = 0$ la ecuación cambia un poco, y se le atribuye el nombre de \textbf{series de MacLaurin} \citep{Stewart2012_Una}.

\begin{equation}
	\boxed{
		f(x) = \sum_{n=0}^{\infty} \frac{f^{(n)}(0)}{n!} x^n = f(0) + \frac{f'(0)}{1!} x + \frac{f''(0)}{2!} x^2 + \cdots
	}
	\label{eq:mclaurin}
\end{equation}

\subsubsection{Aproximación de Orden N}

En la práctica, no es posible calcular una serie infinita de términos. Por lo tanto, se utiliza una aproximación de orden $N$, conocida como el \textbf{polinomio de Taylor de orden N} \citep{taylor_rosario_2013}:

\begin{equation}
	\boxed{
		T_N(x) = \sum_{n=0}^{N} \frac{f^{(n)}(a)}{n!} (x - a)^n
	}
	\label{eq:taylor-orden-n}
\end{equation}

El error de esta aproximación está dado por el \textbf{resto de Taylor} o \textbf{término del error}, que representa la diferencia entre la función real y su aproximación polinómica:

\begin{equation}
	R_N(x) = f(x) - T_N(x)
	\label{eq:resto-taylor}
\end{equation}

\subsubsection{Ejemplos Comunes de Series de Taylor}

Algunas funciones tienen series de Taylor bien conocidas alrededor de $a = 0$ (series de MacLaurin) \citep{Stewart2012_Una}:

\begin{align}
	e^x      & = 1 + x + \frac{x^2}{2!} + \frac{x^3}{3!} + \frac{x^4}{4!} + \cdots = \sum_{n=0}^{\infty} \frac{x^n}{n!}                    \label{eq:taylor-exp} \\[1em]
	\sin(x)  & = x - \frac{x^3}{3!} + \frac{x^5}{5!} - \frac{x^7}{7!} + \cdots = \sum_{n=0}^{\infty} \frac{(-1)^n x^{2n+1}}{(2n+1)!}      \label{eq:taylor-sin}  \\[1em]
	\cos(x)  & = 1 - \frac{x^2}{2!} + \frac{x^4}{4!} - \frac{x^6}{6!} + \cdots = \sum_{n=0}^{\infty} \frac{(-1)^n x^{2n}}{(2n)!}          \label{eq:taylor-cos}  \\[1em]
	\ln(1+x) & = x - \frac{x^2}{2} + \frac{x^3}{3} - \frac{x^4}{4} + \cdots = \sum_{n=1}^{\infty} \frac{(-1)^{n+1} x^n}{n} \quad (|x| < 1) \label{eq:taylor-ln}
\end{align}

\subsubsection{Implementación Computacional}

El algoritmo implementado para calcular una serie de Taylor de orden $N$ alrededor del punto $a$ utiliza un enfoque iterativo que calcula las derivadas de forma incremental. El procedimiento es el siguiente:

\begin{enumerate}
	\item Inicializar la serie de Taylor: $T(x) = 0$
	\item Inicializar la función derivada: $f_{\text{deriv}} = f(x)$ (derivada de orden 0)
	\item Para cada término $n = 0, 1, 2, \ldots, N$:
	      \begin{enumerate}
		      \item Evaluar la derivada actual en el punto $a$: $f^{(n)}(a) = f_{\text{deriv}}(a)$
		      \item Calcular el coeficiente: $c_n = \frac{f^{(n)}(a)}{n!}$
		      \item Calcular el término actual:
		            \[
			            t_n = \begin{cases}
				            c_n                 & \text{si } n = 0 \\
				            c_n \cdot (x - a)^n & \text{si } n > 0
			            \end{cases}
		            \]
		      \item Agregar el término a la serie: $T(x) = T(x) + t_n$
		      \item Si $n < N$, calcular la siguiente derivada: $f_{\text{deriv}} = \frac{d}{dx}f_{\text{deriv}}$
	      \end{enumerate}
	\item Simplificar la expresión final: $T_N(x) = \text{simplify}(T(x))$
\end{enumerate}

Este enfoque iterativo es eficiente ya que calcula cada derivada simbólicamente a partir de la anterior, evitando recalcular todas las derivadas desde cero. La implementación utiliza cálculo simbólico de MATLAB para obtener expresiones exactas de las derivadas y del polinomio resultante.

\subsubsection{Código Matlab}

\begin{lstlisting}
format long;
syms x;
sympref("PolynomialDisplayStyle","ascend"); % Imprimir en orden ascendente, es decir, de menor a mayor grado.




% Solicitar entrada al usuario
f_str = input('Ingrese la funcion: f(x) = ', 's');
FUN = str2sym(f_str);

a = input('Ingrese el punto central de la serie (a): ');
N = input('Ingrese el orden de la serie de Taylor (N): ');

% Validar que N sea un entero positivo
if N < 0 || floor(N) ~= N
    error('El orden N debe ser un entero no negativo.');
end

% Obtener la variable simbolica
x_var = symvar(FUN);
if isempty(x_var)
    error('La funcion debe contener al menos una variable.');
end
x = x_var(1);

% Inicializar la serie de Taylor
T = sym(0);
f_deriv = FUN;

% Imprimir encabezado
fprintf('\n%s\n', repmat('=', 1, 90));
fprintf('CALCULO DE SERIE DE TAYLOR\n');
fprintf('%s\n', repmat('=', 1, 90));
fprintf('Funcion: %s\n', char(FUN));
fprintf('Punto central (a): %s\n', char(a));
fprintf('Orden (N): %d\n', N);
fprintf('%s\n', repmat('=', 1, 90));

% Imprimir tabla de terminos
fprintf('\n%-10s %-30s %-30s %-30s\n', 'Termino', 'f^(n)(a)', 'Coeficiente [ f^(n)(a)/n! ] ', 'Termino de la serie');
fprintf('%s\n', repmat('-', 1, 105));

% Calcular la serie: sum_{n=0}^{N} f^(n)(a)/n! * (x-a)^n
for n = 0:N
        % Evaluar la n-esima derivada en el punto a
        f_at_a = subs(f_deriv, x, a);

        % Calcular el coeficiente
    coef = f_at_a / factorial(n);

    % Calcular el termino actual
    if n == 0
        term = coef;
    else
        term = coef * (x - a)^n;
    end

    % Agregar el termino a la serie
    T = T + term;

    % Imprimir informacion del termino
    fprintf('%-10d %-30s %-30s %-30s\n', ...
        n, ...
        char(vpa(f_at_a, 6)), ...
        char(vpa(coef, 6)), ...
        char(term));

    % Calcular la siguiente derivada para la proxima iteracion
    if n < N
        f_deriv = diff(f_deriv, x);
    end
end

fprintf('%s\n', repmat('-', 1, 105));

% Simplificar la serie final
T = simplify(T);

% Mostrar resultado final
fprintf('\n%s\n', repmat('=', 1, 90));
fprintf('RESULTADO FINAL\n');
fprintf('%s\n', repmat('=', 1, 90));
fprintf('Serie de Taylor de orden %d:\n\n', N);
fprintf('\x1b[32mT_%d(x) = %s\x1b[0m\n', N, char(T));
fprintf('\n%s\n', repmat('=', 1, 90));
\end{lstlisting}

\newpage
\subsubsection{Probando ecuaciones conocidas}

\textbf{Probando con $e^x$ centrado en 0 y de grado 5}: el resultado de la ejecución puede verse en la figura \ref{fig:taylor-exp}, así como su graphica en la figura \ref{fig:graph-taylor-exp}

\begin{figure}[H]
	\centering
	\includegraphics[scale=0.35]{resources/01-taylor/01-exe-exp(x).png}
	\caption{Serie de Taylor de $e^x$ alrededor de $x=0$ con orden 5}
	\label{fig:taylor-exp}
\end{figure}

\begin{figure}[H]
	\centering
	\includegraphics[scale=0.6]{resources/01-taylor/01-graph-exp(x).png}
	\caption{Comparación entre la serie de Taylor alrededor de $0$ de orden 5 y la función original $e^x$}
	\label{fig:graph-taylor-exp}
\end{figure}

\textbf{Probando con $sin(2x)$ centrado en $\frac{\pi}{2}$ y de grado 8}:

\begin{figure}[H]
	\centering
	\includegraphics[scale=0.35]{resources/01-taylor/02-exec-sin(2x).png}
	\caption{Serie de Taylor de $e^x$ alrededor de $x=0$ con orden 5}
	\label{fig:taylor-sin-exec}
\end{figure}

\begin{figure}[H]
	\centering
	\includegraphics[scale=0.6]{resources/01-taylor/02-graph-sin(2x).png}
	\caption{Comparación entre la serie de Taylor alrededor de $0$ de orden 5 y la función original $e^x$}
	\label{fig:taylor-sin-graph}
\end{figure}


\input{content/02-biseccion.tex}
\input{content/03-newton-raphson}

\section{Comparación entre el método de bisección y de Newton-Raphson}

Según \cite{Hussain2024}, el método de Newton-Raphson es considerado eficiente para encontrar raíces de funciones. El mismo autor señala que el método de bisección es más primitivo y lento en comparación con Newton-Raphson. Los autores nos brindan algunos ejemplos que podemos usar para probar nuestros algoritmos.

\subsection{Ejemplo 1: $x^2 + x - 5$}

Para el presente ejemplo vamos a otorgar una tolerancia de error del 0.3\% para cada algoritmo. El periodo será de $[1,2]$


\subsubsection{Ejecución con bisección}

\begin{figure}[H]
	\centering
	\includegraphics[scale=0.35]{resources/05-roots-methos/bisection-coparison.png}
	\caption{Prueba del algoritmo de bisección con la función $x^2 + x - 5$}
	\label{fig:bisection-test}
\end{figure}


\subsubsection{Ejecución con Newton-Raphson}
\begin{figure}[H]
	\centering
	\includegraphics[scale=0.4]{resources/05-roots-methos/newton-comparison.png}
	\caption{Prueba del algoritmo de Newton-Raphson con la función $x^2 + x - 5$}
	\label{fig:newton-test}
\end{figure}


\subsubsection{Conclusiones}

Cómo podemos ver, el algoritmo de Newton-Raphson acaba con considerablemente menos iteraciones que el algoritmo de biseción. Sin embargo, el algoritmo de Newton-Raphson no es siempre confiable, según \cite{Stewart2012_Una}, dependiendo de la función, la siguiente aproximación puede incluso alejarse de la raíz, y en el peor de de los casos, caer fuera del dominio de una función.

Aunque no es una fuente confiable y respaldada por pares, vale la pena al menos leer la siguiente comparación entre algoritmos que da el usuario \textbf{Paul Sinclair} en Stack Exchange en el siguiente enlace: \url{https://math.stackexchange.com/questions/1464795/what-are-the-difference-between-some-basic-numerical-root-finding-methods}

El usuario menciona que:

\begin{itemize}
	\item \textbf{Del algoritmo de bisección}: La \textbf{convergencia está garantizada}, siempre que pueda ``identificar'' la raíz al comienzo. Es fácil de entender, programar y realizar, pero \textit{lento como el infierno}. Nunca envía su iteración al infinito. Este es su método de \textbf{último recurso} cuando todo lo demás falla.

	\item \textbf{Del algoritmo de Newton-Raphson:}  es excelente por su \textbf{velocidad}, pero requiere conocer la \textbf{derivada} (lo cual raramente se encuentra en aplicaciones reales). El principal problema es la \textbf{inestabilidad}: si la función es plana, puede enviar la siguiente iteración \textit{más allá de Plutón}, y no hay \textbf{garantía de convergencia}, pudiendo quedar atrapado en un ciclo.
\end{itemize}



% AJUSTE DE CURVAS 

\section{Ajuste de curvas}

\subsection{Regresión lineal}

Según \cite{Statistical-learning} la regresión lineal es un enfoque muy sencillo para el aprendizaje supervisado. Según los autores, la regresión lineal es, en particular, una herramienta útil para predecir una respuesta cuantitativa.

Según \cite{Xiaogang2012}, el método más común para hallar una regresión lineal es el método de los mínimos cuadrados. La línea que describe el modelo se define, según el autor, como sigue:

%%%%%%%%%%%%


Consideremos el modelo de regresión lineal simple, basandonos en el trabajo de \cite{Statistical-learning}:
\[ y_i = a + b x_i + \epsilon_i \]
donde:
\begin{itemize}
	\item $y_i$ es el valor observado de la variable respuesta para la i-ésima observación
	\item $x_i$ es el valor de la variable predictora para la i-ésima observación
	\item $a$ es el intercepto (término constante)
	\item $b$ es la pendiente (coeficiente de la variable predictora)
	\item $\epsilon_i$ es el error aleatorio
\end{itemize}

El método de mínimos cuadrados minimiza la suma de los cuadrados de los residuos:
\[ Q(a, b) = \sum_{i=1}^n (y_i - a - b x_i)^2 \]

\textbf{Derivación del Intercepto ($a$)}

Calculamos la derivada parcial de $Q$ con respecto a $a$:
\[ \frac{\partial Q}{\partial a} = \frac{\partial}{\partial a} \sum_{i=1}^n (y_i - a - b x_i)^2 \]
\[ = \sum_{i=1}^n 2(y_i - a - b x_i) \cdot (-1) = -2 \sum_{i=1}^n (y_i - a - b x_i) \]

Igualamos a cero:
\[ -2 \sum_{i=1}^n (y_i - a - b x_i) = 0 \]
\[ \sum_{i=1}^n (y_i - a - b x_i) = 0 \]

Expandimos la suma:
\[ \sum_{i=1}^n y_i - \sum_{i=1}^n a - b \sum_{i=1}^n x_i = 0 \]

Dado que $\sum_{i=1}^n a = n a$, tenemos:
\[ \sum_{i=1}^n y_i - n a - b \sum_{i=1}^n x_i = 0 \]

Despejando $a$:
\[ n a = \sum_{i=1}^n y_i - b \sum_{i=1}^n x_i \]
\textcolor{mygreen}{
	\[ a = \frac{1}{n} \left( \sum_{i=1}^n y_i - b \sum_{i=1}^n x_i \right) \]
}
\textbf{Derivación de la Pendiente ($b$)}

Calculamos la derivada parcial de $Q$ con respecto a $b$:
\[ \frac{\partial Q}{\partial b} = \frac{\partial}{\partial b} \sum_{i=1}^n (y_i - a - b x_i)^2 \]
\[ = \sum_{i=1}^n 2(y_i - a - b x_i) \cdot (-x_i) = -2 \sum_{i=1}^n x_i (y_i - a - b x_i) \]

Igualamos a cero:
\[ -2 \sum_{i=1}^n x_i (y_i - a - b x_i) = 0 \]
\[ \sum_{i=1}^n x_i (y_i - a - b x_i) = 0 \]

Expandimos la suma:
\[ \sum_{i=1}^n x_i y_i - a \sum_{i=1}^n x_i - b \sum_{i=1}^n x_i^2 = 0 \]

Sustituimos la expresión de $a$ obtenida anteriormente:
\[ a = \frac{1}{n} \left( \sum_{i=1}^n y_i - b \sum_{i=1}^n x_i \right) \]

Insertamos en la ecuación:
\[ \sum_{i=1}^n x_i y_i - \left( \frac{1}{n} \left( \sum_{i=1}^n y_i - b \sum_{i=1}^n x_i \right) \right) \sum_{i=1}^n x_i - b \sum_{i=1}^n x_i^2 = 0 \]

Multiplicamos por $n$ para eliminar denominadores:
\[ n \sum_{i=1}^n x_i y_i - \left( \sum_{i=1}^n y_i - b \sum_{i=1}^n x_i \right) \sum_{i=1}^n x_i - b n \sum_{i=1}^n x_i^2 = 0 \]

Expandimos:
\[ n \sum_{i=1}^n x_i y_i - \sum_{i=1}^n y_i \sum_{i=1}^n x_i + b \left( \sum_{i=1}^n x_i \right)^2 - b n \sum_{i=1}^n x_i^2 = 0 \]

Reorganizamos términos con $b$:
\[ n \sum_{i=1}^n x_i y_i - \sum_{i=1}^n y_i \sum_{i=1}^n x_i + b \left[ \left( \sum_{i=1}^n x_i \right)^2 - n \sum_{i=1}^n x_i^2 \right] = 0 \]

Despejamos $b$:
\[ b \left[ \left( \sum_{i=1}^n x_i \right)^2 - n \sum_{i=1}^n x_i^2 \right] = \sum_{i=1}^n y_i \sum_{i=1}^n x_i - n \sum_{i=1}^n x_i y_i \]

Multiplicamos por $-1$:
\[ b \left[ n \sum_{i=1}^n x_i^2 - \left( \sum_{i=1}^n x_i \right)^2 \right] = n \sum_{i=1}^n x_i y_i - \sum_{i=1}^n y_i \sum_{i=1}^n x_i \]

Finalmente:
\textcolor{mygreen}{%
	\[ b = \frac{ n \sum_{i=1}^n x_i y_i - \sum_{i=1}^n x_i \sum_{i=1}^n y_i }{ n \sum_{i=1}^n x_i^2 - \left( \sum_{i=1}^n x_i \right)^2 } \]
}

Con base en los coeficientes $a$ y $b$, podemos modelar el comportamiento del dataset y generar así una curva predictora.

\subsubsection{Código matlab}

\begin{lstlisting}
    function linearRegression()
        format long;

        % Ingrese los datos en el formato [x1, y1; x2, y2; ...]
        disp('Ingrese los datos en el formato [x1, y1; x2, y2; ...]:');
        data = input('Datos = ');

        % Extraer las coordenadas x e y de los datos
        x = data(:, 1);
        y = data(:, 2);

        % Numero de puntos de datos
        n = length(x);

        % Calcular las sumas necesarias para el ajuste lineal
        sumX = sum(x);
        sumY = sum(y);
        sumXY = sum(x .* y); % "Element wise operation, ie. x1 * y1 + x2 * y2 + x3 * y3 + ...
        sumX2 = sum(x .^ 2); % Element wise operation, eleva cada termino de la matriz al cuadrado.

        % Calcular los coeficientes de la regresion lineal (y = mx + b)
        m = (n * sumXY - sumX * sumY) / (n * sumX2 - sumX ^ 2);
        b = (sumY - m * sumX) / n;

        % Mostrar los coeficientes de la regresion lineal
        disp('Coeficientes de la regresion lineal:');
        disp(['Pendiente (m): ' num2str(m)]);
        disp(['Intercepto (b): ' num2str(b)]);
       fprintf('\033[32mLa funcion lineal ajustada es: y = %.4fx + %.4f\033[0m\n', m, b);

        % Graficar los datos y la linea de regresion lineal
        fplot(@(x) m * x + b, [min(x) max(x)]);
        hold on;
        scatter(x, y, 'r', 'filled');
        title('Ajuste Lineal (Regresion Lineal)');
        xlabel('x');
        ylabel('y');
        legend('Regresion Lineal', 'Datos', 'Location', 'Best');
        grid on;
        hold off;
    end

    linearRegression()
\end{lstlisting}

\subsection{Regresión cuadrática}

La regresión cuadrática hace parte de lo que se conoce como regresiones polinomiales. Según \cite{Statistical-learning} la manera de representar un modelo  polinomial es como sigue:

\[ y_i = \beta_0 + \beta_1 x_i + \beta_2 x_i^2 + \beta_3 x_i^3 + \ldots + \beta_d x_i^d + \epsilon_i \]

La regresión polinómica permite modelar relaciones no lineales entre una variable dependiente \(y\) y una variable independiente \(x\) \citep{Statistical-learning}. Esto se logra añadiendo términos polinómicos (como \(x^2\), \(x^3\), etc.) a la ecuación.

Así pues, la regresión cuadrática es tan solo una regresión polinómica de grado 2, es decir, incluye un término cuadrático \(x^2\).



\subsubsection{Código matlab}

\begin{lstlisting}
% Ajuste de curvas cuadratico
% NOTA!!!!!! -> Es mejor dar la ruta absoluta del CSV para evitar errores
% cd /Users/esteban/git-repos/numerical-analisis-project && ./run_matlab.sh src/methods/ajuste-de-curvas/cuadratico.m

function quadraticRegression(filename, col_x, col_y)
% quadraticRegression - Realiza regresion cuadratica sobre datos de un archivo CSV
%
% Sintaxis: quadraticRegression(filename, col_x, col_y)
%
% Argumentos:
%   filename - Ruta al archivo CSV (relativa a la raiz del proyecto)
%   col_x    - Numero de columna para la variable independiente (x)
%   col_y    - Numero de columna para la variable dependiente (y)
%
% Ejemplo:
%   quadraticRegression('guia/datos.csv', 6, 7)
format long;

% Obtener el directorio del script y cambiar al directorio del proyecto
scriptPath = fileparts(mfilename('fullpath'));
projectPath = fullfile(scriptPath, '..', '..', '..');
cd(projectPath);
disp(['Directorio de trabajo: ' pwd]);

% Leer el archivo CSV
try
opts = detectImportOptions(filename, 'Delimiter', ';', 'DecimalSeparator', ',');
% Leer encabezados primero para mostrar columnas disponibles
headers = readtable(filename, opts, 'ReadRowNames', false);
disp('Archivo cargado exitosamente.');
disp(['Columnas disponibles: ' strjoin(headers.Properties.VariableNames, ', ')]);

% Obtener nombres de las columnas seleccionadas
column_names = headers.Properties.VariableNames;
x_label = strrep(column_names{col_x}, '_', ' ');
y_label = strrep(column_names{col_y}, '_', ' ');

% Ahora leer solo los datos numericos
data = readmatrix(filename, 'Delimiter', ';', 'DecimalSeparator', ',');
disp(['Numero de filas: ' num2str(size(data, 1))]);
catch ME
disp(['Error: ' ME.message]);
error('No se pudo leer el archivo. Verifique la ruta y el formato.');
end

% Extraer las coordenadas x e y de las columnas especificadas
x = data(:, col_x);
y = data(:, col_y);

% Verificar que los datos sean numericos
if ~isnumeric(x) || ~isnumeric(y)
error('Las columnas seleccionadas deben contener datos numericos.');
end

% Eliminar filas con valores NaN
valid_idx = ~isnan(x) & ~isnan(y);
x = x(valid_idx);
y = y(valid_idx);

% Numero de puntos de datos
n = length(x);
disp(['Numero de puntos validos: ' num2str(n)]);

% Validar que haya al menos 3 puntos para la regresion cuadratica
if n <= 2
error('Se requieren al menos 3 puntos para la regresion cuadratica.');
end

% Realizar regresion cuadratica (ajustar un polinomio de grado 2)
% polyfit retorna los coeficientes [a, b, c] para ax^2 + bx + c
p = polyfit(x, y, 2);

% Mostrar los coeficientes del polinomio cuadratico
disp('Coeficientes del polinomio cuadratico:');
disp(['a (coef. x^2): ' num2str(p(1))]);
disp(['b (coef. x): ' num2str(p(2))]);
disp(['c (intercepto): ' num2str(p(3))]);

% Imprimir la ecuacion final en color verde usando codigos ANSI
% Formato: f(x) = ax^2 + bx + c
fprintf('\033[32mLa funcion cuadratica ajustada es: y = %.4fx^2 + %.4fx + %.4f\033[0m\n', p(1), p(2), p(3));

% Generar puntos para graficar la curva ajustada
x_fit = linspace(min(x), max(x), 100);  % Crear 100 puntos equiespaciados entre el x minimo y maximo
y_fit = polyval(p, x_fit);               % Evaluar el polinomio en estos puntos

% Graficar los puntos de datos originales como circulos rellenos
scatter(x, y, 80, 'b', 'filled', 'MarkerEdgeColor', 'k', 'LineWidth', 1, 'MarkerFaceAlpha', 0.6);
hold on;  % Mantener el grafico para agregar mas elementos

% Graficar la curva cuadratica ajustada como una linea roja
plot(x_fit, y_fit, 'r-', 'LineWidth', 2);

% Agregar etiquetas y formato al grafico usando los nombres de las columnas
xlabel(x_label, 'FontSize', 12, 'FontWeight', 'bold');
ylabel(y_label, 'FontSize', 12, 'FontWeight', 'bold');
title(['Regresion Cuadratica: ' y_label ' vs ' x_label], 'FontSize', 14, 'FontWeight', 'bold');
legend('Datos', sprintf('y = %.4fx^2 + %.4fx + %.4f', p(1), p(2), p(3)), 'Location', 'best');
grid on;   % Mostrar lineas de cuadricula
hold off;  % Liberar el grafico
end
quadraticRegression('/Users/esteban/git-repos/numerical-analisis-project/guia/datos.csv', 6, 7)
\end{lstlisting}



\subsubsection{Haciendo uso del modelo con el dataset de prueba}

Para ilustrar el modelo cuadrático, utilizaremos el dataset mostrado anteriomente en la figura \ref{fig:dataset}.


\subsubsection{Modelo Cuadrático: Notas de algoritmos VS Hora de estudio}


Haciendo uso del script de matlab elaborado para la presente sección, que el modelo produce la curva $-0.0003x^2 + 0.0400x + 3.1257$ como se puede observar en la figura \ref{fig:ajuste-cuadratico}.



\begin{figure}[H]
	\centering
	\includegraphics[scale=0.5]{resources/ajuste-curvas/cuadratico_horas_notas.png}
	\caption{Ejecución del ajuste cuadrático del modelo Horas de Estudio VS Notas de Algoritmos}
	\label{fig:ajuste-cuadratico}
\end{figure}


La gráfica del modelo cuadrático se puede ver a continuación en la figura \ref{fig:plot:ajuste-cuadratico}.



\begin{figure}[H]
	\centering
	\includegraphics[scale=0.4]{resources/ajuste-curvas/plot_cuadratico_horas_notas.png}
	\caption{Gráfica del ajuste cuadrático por mínimos cuadrados del modelo Horas de Estudio VS Notas de Algoritmos}
	\label{fig:plot:ajuste-cuadratico}
\end{figure}







% Bibliografía
\newpage
\bibliographystyle{apalike-impure}
\bibliography{references.bib}

\end{document}
