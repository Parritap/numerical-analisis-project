\documentclass[12pt,a4paper]{article}

% Paquetes para el idioma español
\usepackage[spanish,es-tabla]{babel}

%% Configuración de fuente Arial
%\usepackage{fontspec}
%\setmainfont{Arial}

% Paquetes adicionales
\usepackage{geometry}
\usepackage{setspace}
\usepackage{graphicx}
\usepackage{amsmath}
\usepackage{amsfonts}
\usepackage{amssymb}
\usepackage{url}
\usepackage{hyperref}
\usepackage[authoryear,round]{natbib}
\usepackage[table]{xcolor}
\usepackage{colortbl}
\usepackage{array}
\usepackage{longtable}
\usepackage{float}

% Definir colores para tablas
\definecolor{HeaderGreen}{RGB}{23,162,153}
\definecolor{LightGreen}{RGB}{204,227,220}

% Configuración de natbib para español
\providecommand{\andname}{and}
\renewcommand{\andname}{y}
\addto\captionsspanish{%
  \renewcommand{\bibname}{Bibliograf\'ia}%
  \renewcommand{\refname}{Referencias}%
}




% Configuración de márgenes
\geometry{
    left=2.5cm,
    right=2.5cm,
    top=2.5cm,
    bottom=2.5cm
}


% Configuración de interlineado
\onehalfspacing

% Configuración de hipervínculos
\hypersetup{
    colorlinks=true,
    linkcolor=black,
    citecolor=blue,
    urlcolor=blue,
    pdftitle={Título del Documento},
    pdfauthor={Autor},
    pdfsubject={Materia},
    pdfkeywords={palabras clave}
}

% Información del documento
\title{Métodos numéricos en Matlab}
\author{Juan Esteban Parra Parra}
\date{\today}


\begin{document}


% Página de título
\begin{titlepage}
	\centering


	\vspace{2em}
	{\textbf{Métodos numéricos en Matlab}}

	{Profesor:}\\
	\vspace{0.1cm}
	{Edwin Romero Cuero}\\


	\vspace{1.2cm}

	% Logo centrado como primer elemento
	\begin{center}
		\includegraphics[width=0.4\textwidth]{resources/uniquindio.png}
	\end{center}


	\vspace{1.1cm}
	% ####### PRESENTACION 
	{\textbf{Presentado por:}}\\
	\vspace{0.1cm}
	{Juan Esteban Parra Parra}\\

	\vfill{

		\vspace{1cm}
		\vspace{1cm}

		{Universidad del Quindío --- Facultad de ingeniería\\Programa de Ingeniería de Sistemas y Computación}\\
		\vspace{0.5cm}
		{Armenia -- Quindío 2025 }



	}

\end{titlepage}

% Tabla de contenidos
\newpage
\tableofcontents
\newpage

\section{Objetivo}
Desarrollar e implementar funciones en MATLAB que apliquen los métodos numéricos vistos en clase para resolver problemas de ingeniería, enseñados en la asignatura Análisis numérico y que nos proporciona el docente Edwin Romero Cuero.
\noindent
Buscamos a través de esta actividad, fortalecer la comprensión de los distintos métodos de análisis aprendidos durante este semestre, a su vez comprender los errores de truncamiento, técnicas de búsqueda de raíces, regresiones, interpolación y ecuaciones diferenciales.

% Contenido principal del documento
\section{Introducción}

En esta actividad se desea utilizar varios métodos numéricos y aplicarlos en un sistema de cómputo numérico “MATLAB”, buscando darle una solución a problemas que son comúnmente vistos en el ámbito de la ingeniería, donde se tratan temas como series de Taylor, métodos de búsqueda de raíces, regresión, interpolación y el uso de los métodos de Euler y Runge-Kutta para resoluciones numéricas de ecuaciones diferenciales. Mediante el uso de los conocimientos apropiados en el ambiente de desarrollo se planea poner en pruebas los conocimientos teóricos con problemas recurrentes en el área laboral.


\section{Métodos para encontrar raices}

\subsection{Bisección}

Según \cite{wikipedia_biseccion}, el método de bisección, también llamado dicotomía, es un algoritmo de búsqueda de raíces que trabaja dividiendo el intervalo a la mitad y seleccionando el subintervalo que tiene la raíz.

\section{Conclusiones}

Conclusiones del trabajo.

% Bibliografía
\newpage

\bibliographystyle{apalike-impure}
\bibliography{references}

\end{document}
