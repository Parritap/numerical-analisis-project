%% %%%%%%%%%%%%%%%%%%%%%%%%%%%%%%%%%%%%%%%%%%%%%%%%%
%% Template for a conference paper, prepared for the
%% Food and Resource Economics Department - IFAS
%% UNIVERSITY OF FLORIDA
%% %%%%%%%%%%%%%%%%%%%%%%%%%%%%%%%%%%%%%%%%%%%%%%%%%
%% Version 1.0 // November 2019
%% %%%%%%%%%%%%%%%%%%%%%%%%%%%%%%%%%%%%%%%%%%%%%%%%%
%% Ariel Soto-Caro
%%  - asotocaro@ufl.edu
%%  - arielsotocaro@gmail.com
%% %%%%%%%%%%%%%%%%%%%%%%%%%%%%%%%%%%%%%%%%%%%%%%%%%
\documentclass[11pt]{article}
\usepackage{UF_FRED_paper_style}

\usepackage{lipsum}  %% Package to create dummy text (comment or erase before start)

%% ===============================================
%% Setting the line spacing (3 options: only pick one)
% \doublespacing
% \singlespacing
\onehalfspacing
%% ===============================================

\setlength{\droptitle}{-5em} %% Don't touch

% %%%%%%%%%%%%%%%%%%%%%%%%%%%%%%%%%%%%%%%%%%%%%%%%%%%%%%%%%%
% SET THE TITLE
% %%%%%%%%%%%%%%%%%%%%%%%%%%%%%%%%%%%%%%%%%%%%%%%%%%%%%%%%%%

% TITLE:
\title{Proyecto Final Análisis Numérico}

% AUTHORS:
\author{First Author\\% Name author
    \href{mailto:juane.parrap@uqvirtual.edu.co}{\texttt{juane.parrap@uqvirtual.edu.co}} %% Email author 1 
\and Second Author\\% Name author
    \href{mailto:secondauthor@ufl.edu}{\texttt{davidc.rojasX@uqvirtual.edu.co}} %% Email author 2
   }
    
% DATE:
\date{\today}

% %%%%%%%%%%%%%%%%%%%%%%%%%%%%%%%%%%%%%%%%%%%%%%%%%%%%%%%%%%
% %%%%%%%%%%%%%%%%%%%%%%%%%%%%%%%%%%%%%%%%%%%%%%%%%%%%%%%%%%
\begin{document}
% %%%%%%%%%%%%%%%%%%%%%%%%%%%%%%%%%%%%%%%%%%%%%%%%%%%%%%%%%%
% %%%%%%%%%%%%%%%%%%%%%%%%%%%%%%%%%%%%%%%%%%%%%%%%%%%%%%%%%%
% ABSTRACT
% %%%%%%%%%%%%%%%%%%%%%%%%%%%%%%%%%%%%%%%%%%%%%%%%%%%%%%%%%%
% %%%%%%%%%%%%%%%%%%%%%%%%%%%%%%%%%%%%%%%%%%%%%%%%%%%%%%%%%%
{\setstretch{.8}
\maketitle
% %%%%%%%%%%%%%%%%%%
\begin{abstract}
	% CONTENT OF ABS HERE--------------------------------------

	This activity aims to employ various numerical methods and implement them within a numerical computing environment such as MATLAB. The objective is to address problems commonly encountered in the field of engineering, involving topics such as Taylor series, root-finding methods, regression, interpolation, and the application of Euler and Runge-Kutta methods for the numerical solution of differential equations.
	By leveraging appropriate knowledge within the development environment, the intention is to test theoretical concepts through the resolution of practical problems frequently observed in professional settings.

	% END CONTENT ABS------------------------------------------
	\noindent
	\textit{\textbf{Keywords: }%
		Análisis numérico; Matlab; Computación numérica; Algoritmos.} \\ %% <-- Keywords HERE!
	\noindent

\end{abstract}
}

% %%%%%%%%%%%%%%%%%%%%%%%%%%%%%%%%%%%%%%%%%%%%%%%%%%%%%%%%%%
% %%%%%%%%%%%%%%%%%%%%%%%%%%%%%%%%%%%%%%%%%%%%%%%%%%%%%%%%%%
% BODY OF THE DOCUMENT
% %%%%%%%%%%%%%%%%%%%%%%%%%%%%%%%%%%%%%%%%%%%%%%%%%%%%%%%%%%
% %%%%%%%%%%%%%%%%%%%%%%%%%%%%%%%%%%%%%%%%%%%%%%%%%%%%%%%%%%

% --------------------
\section{Introduction}
% --------------------

En esta actividad se desea utilizar varios métodos numéricos y aplicarlos en un sistema de cómputo numérico  “MATLAB”, buscando darle una solución a problemas que son comúnmente vistos en el ámbito de la ingeniería, donde se tratan temas como series de Taylor, métodos de búsqueda de raíces, regresión, interpolación y el uso de los métodos de Euler y Runge-Kutta para resoluciones numéricas de ecuaciones diferenciales.
Mediante el uso de los conocimientos apropiados en el ambiente de desarrollo se planea poner en pruebas los conocimientos teóricos con problemas recurrentes en el área laboral.


% --------------------
\section{Methodology}
% --------------------

\lipsum[7-9] % Dummy text. Erase before write
\citep{Chavas2015} % Example of citation. Erase before use

% --------------------
\section{Results}
% --------------------

\lipsum[12-13] % Dummy text. Erase before write

% --------------------
\section{Discussion and Conclusions}
% --------------------

\lipsum[14] % Dummy text. Erase before write

% %%%%%%%%%%%%%%%%%%%%%%%%%%%%%%%%%%%%%%%%%%%%%%%%%%%%%%%%%%
% %%%%%%%%%%%%%%%%%%%%%%%%%%%%%%%%%%%%%%%%%%%%%%%%%%%%%%%%%%
% REFERENCES SECTION
% %%%%%%%%%%%%%%%%%%%%%%%%%%%%%%%%%%%%%%%%%%%%%%%%%%%%%%%%%%
% %%%%%%%%%%%%%%%%%%%%%%%%%%%%%%%%%%%%%%%%%%%%%%%%%%%%%%%%%%
\medskip

\bibliography{references.bib}

\newpage

% %%%%%%%%%%%%%%%%%%%%%%%%%%%%%%%%%%%%%%%%%%%%%%%%%%%%%%%%%%
% %%%%%%%%%%%%%%%%%%%%%%%%%%%%%%%%%%%%%%%%%%%%%%%%%%%%%%%%%%
% TABLES
% %%%%%%%%%%%%%%%%%%%%%%%%%%%%%%%%%%%%%%%%%%%%%%%%%%%%%%%%%%
% %%%%%%%%%%%%%%%%%%%%%%%%%%%%%%%%%%%%%%%%%%%%%%%%%%%%%%%%%%

\begin{table}[H]
	\centering
	\caption{Example table of descriptive statistics of the main variables.}
	\label{tab:1}
	\scalebox{.8}{
		\begin{tabular}{rlrrrrrr}
			\hline
			\multicolumn{1}{c}{\textbf{Variables}} & \multicolumn{1}{c}{\textbf{Categories}} & \multicolumn{1}{c}{\textbf{Unit}} & \multicolumn{1}{c}{\textbf{Rep}} & \multicolumn{1}{c}{\textbf{Mean}} & \multicolumn{1}{c}{\textbf{St. Dev.}} & \multicolumn{1}{c}{\textbf{Min}} & \multicolumn{1}{c}{\textbf{Max}} \\ \hline \hline

			\multicolumn{1}{l}{Variable 1}         & Category A                              & \multicolumn{1}{c}{\$}            & \multicolumn{1}{c}{8}            & 0                                 & 0                                     & 0                                & 0                                \\
			                                       & Category B                              & \multicolumn{1}{c}{lb}            & \multicolumn{1}{c}{8}            & 22,411.20                         & 6,325.90                              & 13,819                           & 31,201                           \\
			                                       & Category C                              & \multicolumn{1}{c}{\$}            & \multicolumn{1}{c}{8}            & 5,869.60                          & 4,609.90                              & -464.1                           & 12,744.10                        \\
			\multicolumn{1}{l}{Variable 2}         & Category A                              & \multicolumn{1}{c}{\$}            & \multicolumn{1}{c}{8}            & 1,777.40                          & 144.5                                 & 1,642.30                         & 1,912.60                         \\
			                                       & Category B                              & \multicolumn{1}{c}{lb}            & \multicolumn{1}{c}{8}            & 21,444.80                         & 5,146.90                              & 15,096                           & 28,032                           \\
			                                       & Category C                              & \multicolumn{1}{c}{\$}            & \multicolumn{1}{c}{8}            & 4,138.50                          & 2,644.10                              & 22.2                             & 7,932.70                         \\
			\multicolumn{1}{l}{Variable 3}         & Category A                              & \multicolumn{1}{c}{\$}            & \multicolumn{1}{c}{8}            & 2,346.80                          & 190.8                                 & 2,168.30                         & 2,525.20                         \\
			                                       & Category B                              & \multicolumn{1}{c}{lb}            & \multicolumn{1}{c}{8}            & 18,343.30                         & 2,460.70                              & 15,269.00                        & 21,524.10                        \\
			                                       & Category C                              & \multicolumn{1}{c}{\$}            & \multicolumn{1}{c}{8}            & 3,699.20                          & 2,549.80                              & 1,299.10                         & 8,709.80                         \\
			\multicolumn{1}{l}{Variable 4}         & Category A                              & \multicolumn{1}{c}{\$}            & \multicolumn{1}{c}{8}            & 2,288.80                          & 186.1                                 & 2,114.80                         & 2,462.90                         \\
			                                       & Category B                              & \multicolumn{1}{c}{lb}            & \multicolumn{1}{c}{8}            & 23,450.40                         & 4,172.50                              & 20,045.00                        & 32,363.00                        \\
			                                       & Category C                              & \multicolumn{1}{c}{\$}            & \multicolumn{1}{c}{8}            & 6,619.80                          & 1,918.40                              & 4,479.70                         & 10,633.90                        \\
			\hline
			                                       & CASE \#1                                &                                   &                                  & 14                                & 6.61                                  & 6.9                              & 27.9                             \\
			                                       & CASE \#2                                &                                   &                                  & 22.8                              & 7.73                                  & 10.2                             & 31.4                             \\
			\hline
		\end{tabular}}
\end{table}%


% %%%%%%%%%%%%%%%%%%%%%%%%%%%%%%%%%%%%%%%%%%%%%%%%%%%%%%%%%%
% %%%%%%%%%%%%%%%%%%%%%%%%%%%%%%%%%%%%%%%%%%%%%%%%%%%%%%%%%%
% FIGURES
% %%%%%%%%%%%%%%%%%%%%%%%%%%%%%%%%%%%%%%%%%%%%%%%%%%%%%%%%%%
% %%%%%%%%%%%%%%%%%%%%%%%%%%%%%%%%%%%%%%%%%%%%%%%%%%%%%%%%%%
%
% \begin{figure}[H]
% 	\centering
% 	\includegraphics[scale=.6]{figures/example_figure.png}
% 	\caption{Example figure.}
% 	\label{fig:1}
% \end{figure}
%
% ==========================
% ==========================
% ==========================


\end{document}
